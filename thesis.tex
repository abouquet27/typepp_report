%%%%%%%%%%%%%%%%%%%%%%%%%%%%%%%%%%%%%%%%%%%%%%%%%%%%%%%%%%%
% EPFL report package, main thesis file
% Goal: provide formatting for theses and project reports
% Author: Mathias Payer <mathias.payer@epfl.ch>
%
% To avoid any implication, this template is released into the
% public domain / CC0, whatever is most convenient for the author
% using this template.
%
%%%%%%%%%%%%%%%%%%%%%%%%%%%%%%%%%%%%%%%%%%%%%%%%%%%%%%%%%%%
\documentclass[a4paper,11pt,oneside]{report}
% Options: MScThesis, BScThesis, MScProject, BScProject
\usepackage[MScProject,lablogo]{EPFLreport}
\usepackage{xspace}
\usepackage{xcolor}
\usepackage[chapter]{minted}
\setminted{numbersep=3pt}
\setminted{fontsize=\footnotesize}
\usepackage{listings}

\title{Building a robust test suite of type confusion vulnerabilities}
\author{Adrien Bouquet}
\supervisor{Nicolas Badoux}
\adviser{Prof. Dr. sc. ETH Mathias Payer}

\expert{The External Reviewer}

\newcommand{\sysname}{FooSystem\xspace}
\newcommand{\todobox}[3]{%
       \colorbox{#1}{\textcolor{white}{\sffamily\bfseries\scriptsize #2}}%
       ~\textcolor{#1}{#3} %
       \textcolor{#1}{$\triangleleft$}%
}
\newcommand{\nb}[1]{\todobox{violet}{NB}{#1}}
\newcommand{\adrbqt}[1]{\todobox{red}{TODO}{#1}}

\begin{document}
\maketitle
\dedication{\textit{Program testing can be used to show the presence of bugs, but never to show their absence!}
\begin{flushright}
"Notes On Structured Programming" by Edsger W. Dijkstra, April 1970.
\end{flushright}
}
\makededication
\acknowledgments{
I would like to thank my supervisor, Nicolas Badoux, for his guidance and support throughout this project. 
Finally, I would like to thank my family and friends for their support
}
\makeacks



\begin{abstract}

The abstract serves as an executive summary of your project.
Your abstract should cover at least the following topics, 1-2 sentences for
each: what area you are in, the problem you focus on, why existing work is
insufficient, what the high-level intuition of your work is, maybe a neat
design or implementation decision, and key results of your evaluation.

My project: 
Explain Type confusion, and why they are a problem i.e., memory attacks, safety, more control over types
compare two mechanisms to type confusion.
Goal of the paper i.e., creating a general benchmark for type confusion and explain the results


\end{abstract}

\maketoc

%%%%%%%%%%%%%%%%%%%%%%
\chapter{Introduction}
%%%%%%%%%%%%%%%%%%%%%%

The introduction is a longer writeup that gently eases the reader into your
thesis~\cite{dinesh20oakland}. Use the first paragraph to discuss the setting.
In the second paragraph you can introduce the main challenge that you see.
The third paragraph lists why related work is insufficient.
The fourth and fifth paragraphs discuss your approach and why it is needed.
The sixth paragraph will introduce your thesis statement. Think how you can
distill the essence of your thesis into a single sentence.
The seventh paragraph will highlight some of your results
The eights paragraph discusses your core contribution.

My project: 
1. Explain Type confusion, Hextype mechanism and type++ mechanism
2. Measure type++ mechanism by assessing tests and compare it to Hextype
3. Explain the aim of the paper, i.e. general benchmark for type confusion and how he achieves it

%%%%%%%%%%%%%%%%%%%%
\chapter{Background}
%%%%%%%%%%%%%%%%%%%%

The background section introduces the necessary background to understand your
work. This is not necessarily related work but technologies and dependencies
that must be resolved to understand your design and implementation.

This section is usually 3-5 pages.

My project:
0. Introduce C++, Type confusion
1. Explain in more detail and with examples, C++ inheritance and casting
2. Hextype and type++ paradigm and how they handle type confusion

C++ is a general purpose 
programming language widely-used in a lot of domains
such as operating systems, or machine learning. As C++ is a compiled language, it
allows developers to write extremely fast and efficient code
that run with minimal overhead. Types are a core concept in C++
but typecasting, which is the action of casting from a type to another,
is source of vulnerabilities. For instance, type confusion is a common
security issue in C++, which occurs when the compiler misinterprets a pointer
type generally after a wrong typecasting. This can lead to memory corruption
and security vulnerabilities. An adversary can then exploit this vulnerability
to execute arbitrary code, heap corruption and hijack the control-flow.  Such an
attack occurred in the past in V8 in Google Chrome, where it allowed a remote
attacker to potentially exploit heap corruption via a crafted HTML page
(CVE-2023-3079)~\cite{nist}. Hence, it is crucial to detect wrong typecasting which induce
type confusion.  These wrong casts can occur when casting a pointer outside a
type hierarchy (\texttt{char*} to \texttt{void*} then casting to another type,
see line \ref{code:unrelated_type_confusion} in Listing
\autoref{lst:type_confusion}) or inside a type hierarchy with classes and
object-oriented programming e.g., casting a pointer from a base class to a
derived class. This can lead to memory corruption and security
vulnerabilities. In this chapter, we will present the concept of the object-oriented
programming, class hierarchies and inheritance in C++ \autoref{sec:cpp_oop}, and
present the different type casting operators in C++
\autoref{sec:cpp_type_casting}.  Finally, we will present the different
mitigation mechanisms to type confusion and how they handle it \autoref{sec:cpp_mechanisms}.

\begin{listing}
       \begin{minted}[frame=single,linenos, breaklines, escapeinside=||]{C++}
#include <string>
#include <iostream>
using namespace std;

class Base {}; // Base class
class Dummy: public Base { // Derived class
       public:
       virtual void printMessage(string message) {
              cout << message << endl;
       }
       
};

class Password: public Base { // Derived class 2
       string password = "empty";
       // private function that cannot be accessed from outside the class
       virtual void modifyPassword(string newPassword) {
              password = newPassword;
       }
       public:
       Password(string newPassword) { modifyPassword(newPassword); }

       void printPassword() { cout << password << endl; }
};

class Unrelated { // Unrelated class
       public: 
       virtual void doRandomStuffWithString(string str) {
              /* random stuff*/
              cout << str << endl;
       }
};

void typeConfusionInHierarchy(Base* base) { // Type confusion inside the hierarchy
       Dummy* dummy = static_cast<Dummy*>(base); |\label{code:derived_type_confusion}|
       dummy->printMessage("You have been hacked"); // Illegal access |\label{code:illegal_access1}|
}

void typeConfusionOutsideHierarchy(Base* base) { // Type confusion outside the hierarchy
       Unrelated* unrelated = reinterpret_cast<Unrelated*>(base); |\label{code:unrelated_type_confusion}|
       unrelated->doRandomStuffWithString("You have been hacked twice"); // Illegal access |\label{code:illegal_access2}|
}

int main() {
       Password password("MyHardPassword"); // Password initialization |\label{code:password}|
       Base* base = &password; // upcast to base class
       password.printPassword(); // print "MyHardPassword"
       typeConfusionInHierarchy(base); // polymorphism |\label{code:polymorphism}|
       password.printPassword(); // print "You have been hacked"
       typeConfusionOutsideHierarchy(&password);
       password.printPassword(); // print "You have been hacked twice"
       return 0;
}
       \end{minted}
       \caption{Example of type confusion in C++ within the same class hierarchy.}
       \label{lst:type_confusion}
\end{listing}

\section{Object-Oriented Programming, Class Hierarchies and Inheritance}
\label{sec:cpp_oop} Object-Oriented Programming (OOP) is \a concept focus on creating objects
which are instances of classes that contain both data and functions. This
paradigm provides data abstraction, faster execution and code reusability.
Classes are organized in hierarchies, generally with one base class and one
derived, and can inherit attributes and methods from other classes which
they then can use or override. In Listing \autoref{lst:type_confusion}, classes
\texttt{Password} and \texttt{Dummy} are derived from the base class
\texttt{Base}, whereas the class \texttt{Unrelated} is unrelated to their class
hierarchy.  A derived class can have multiple base classes and thus possesses
all the methods and attributes from its base classes.  Furthermore, a derived
class is a subtype of its base class, meaning that a derived class can be used
in place of its base class e.g., at line \ref{code:polymorphism} a
\texttt{Password} pointer is given instead of a \texttt{Base} pointer. This
concept which is called polymorphism, is remarkably practical but leads to some
issues. For instance, when a base class has two derived classes that are not
directly related to each other, it is possible to cast a pointer from one
derived class to another in type casting by the base class. This comes from 
C compatibility as all casts are allowed in C and C++ developers tried 
to keep that. However, this behavior induces type
confusion, as at line \ref{code:derived_type_confusion}, where a
\texttt{Password*} is cast to a \texttt{Dummy*} after being cast to a
\texttt{Base*}. 

\section{Type Casting} \label{sec:cpp_type_casting} Typecasting is the operation
of casting a pointer from a type to another. C++ provides four casting
operators: \texttt{static\_cast}, \texttt{dynamic\_cast}, \texttt{const\_cast}
and \texttt{reinterpret\_cast}. 

\texttt{\textbf{const\_cast}<target-type>(expression)} revokes the \texttt{const} or
\texttt{volatile} attribute of an object. Due to operator's behavior, it does not impact type
hierarchies and thus does not lead to type confusion, although there are some
issues with \texttt{const\_cast}. It is out of scope of this paper and will not 
be discussed further.

\texttt{\textbf{static\_cast}<target-type>(expression)} cast an expression to a
specified type while checking at compile-time if target type and expression's 
type are related but not at runtime. Therefore, this
reduces cast verification to checking the types in the type hierarchy. It does
not guarantee the types are compatible at runtime and places the onus back on
programmers to ensure type correctness. Hence, it leads to a type confusion
vulnerability when misused i.e., casting from the base class to a derived class
or when casting a pointer to a void pointer.

\texttt{\textbf{reinterpret\_cast}<target-type>(expression)} similarly to
static\_cast, it casts an expression to a specified type by reinterpreting the
underlying bit pattern but with the difference that it does not check anything neither at
compile time nor at runtime. Hence, it can potentially cast two incompatible
types and lead to type confusion.  It is also the programmer's responsibility to
ensure the correctness of the target-type and the type of expression.

\texttt{\textbf{dynamic\_cast}<target-type>(expression)} cast safely an
expression to a specified type and checks at compile-time and at runtime, unlike
\texttt{static\_cast}, if the types are compatible. To achieve the runtime
check, some forms of \texttt{dynamic\_cast} rely on run-time type identification (RTTI), 
that is, information about each polymorphic class in the compiled program. 
However, this metadata is only available when the class has at least
one virtual function and thus is polymorphic, which constraint to use
\texttt{dynamic\_cast} only on polymorphic classes.

Finally, C-style typecasting exists in C++ due to backward compatibility between
both languages but highly discouraged due for instance to its lack of clarity
for programmers.  Furthermore, the compiler translate C-style typecasting to (i)
\texttt{const\_cast}, (ii) \texttt{static\_cast} and (iii)
\texttt{reinterpret\_cast} in this preferred order which passes on the
underlying types confusions.

\section{Type confusions and Mechanisms} \label{sec:cpp_mechanisms}

As previously mentioned, type confusion vulnerabilities can lead to some memory
corruption and security vulnerabilities which products such as Telegram
(CVE-2021-31318, CVE- 2021-31317) or Google Chromium
(CVE-2021-30627,CVE-2022-1096, CVE-2022-1134, CVE-2022-2295) are exposed to.
Listing \autoref{lst:type_confusion} shows an example of a type confusion.
First, we define a base class \texttt{Base} and two derived classes
\texttt{Dummy} and \texttt{Password} with their own attributes and methods. We
set up the password to "MyHardPassword" (line \ref{code:password}) of our
class.  A type confusion inside a type hierarchy can occur when there is a
cast from a base to a derived class.  In our example, the password pointer is
cast to its base and passed (line \ref{code:polymorphism}) to
\texttt{typeConfusionInHierarchy()} (line \ref{code:illegal_access1}) which cast
it to a \texttt{Dummy*} (line \ref{code:derived_type_confusion}).  From this
line, the behavior is undefined and can lead to memory corruption. Even if it's
calling \texttt{printMessage()} of \texttt{Dummy}, it will execute
\texttt{modifyPassword()} which will modify our password.  This behavior is due
to object being a \texttt{Password} object, and it will call
\texttt{modifyPassword()} method.  A type confusion outside a type hierarchy
can occur when there is a cast from a base to an unrelated class.  We can
observe this behavior in \texttt{typeConfusionOutsideHierarchy()} (line
\ref{code:illegal_access2})  where the password pointer is cast to an
\texttt{Unrelated*} (line \ref{code:unrelated_type_confusion}).  Calling
\texttt{doRandomStuffWithString()} will call in fact \texttt{modifyPassword()}
and modify our password a second time.  An adversary could exploit
vulnerabilities, especially if the program takes some inputs to the program
from its user.  However, some downcast or unrelated cast be legitimate when the
programmer is sure of the type of the object, and we want to avoid false
positive as well as false negative.

Hence, it is a problem that require to be handled for developer and user safety.
The compiler checks statically the correctness of each cast to ensure that the
types are compatible, but it does not guarantee the same correctness at runtime, 
because for instance an object of a derived class can be used in place of an
object of a base class due to polymorphism and vice versa as in line
\ref{code:polymorphism}.  To detect type confusion at runtime, there exists two
approaches: (i) relying on disjoint metadata, such as type table to keep track
of each type of each object or trees for type hierarchy, and (ii) relying on
existing metadata Runtime Type Information (RTTI), such as vtables. 
Generally, the first approach is more efficient because it covers more types, but
it introduces overhead and performance issues. The second is faster but less
efficient due to its limitation to polymorphic classes, and it caused some crash
issue.  In this paper, we will focus and compare on two methods: Hextype and
type++.  Both checks types correctness at runtime, but they differ in their
approach. Hextype is older and relies on disjoint metadata to achieve types
correctness.  Type++ is a recent C++ dialect and relies on existing metadata to
achieve types correctness but unlike an approach based solely on existing
metadata for polymorphism object, it extends this by adding a vtable pointer to
each object to increase efficiency, cover more types and avoid crash issue. In
the following, we will detail how Hextype and type++ work.  Comparing them
offers a perspective on the effectiveness of each approach, despite the age
difference between the development of these two mechanisms.

\subsection{Hextype}
What I want to talk about Hextype?
\begin{itemize}
       \item What is Hextype?
       \item How does it work?
       \item Its limitations (detect type confusion from cast inside the hierarchy)
\end{itemize}

HexType is a Clang/LLVM-based type confusion sanitizer used to detect type
confusion vulnerabilities in C++ programs.  It relies on disjoint metadata to
keep track of each type of each object and to check types correctness at
runtime.

\subsection{type++}

What I want to talk about type++?
\begin{itemize}
       \item What is type++?
       \item How does it work?
       \item Its limitations \adrbqt{To be discussed with Nicolas}
\end{itemize}


%%%%%%%%%%%%%%%%
\chapter{Design}
%%%%%%%%%%%%%%%%

Type confusion sanitizers are designed to detect type confusion vulnerabilities
in C++ programs.  Our threat model assumes that an adversary can exploit type
confusion vulnerabilities to execute arbitrary code.  We built a test suite with
a multitude of cases to evaluate how a sanitizer or a C++ dialect can detect
these types confusion and, a fortiori, try to prevent such adversaries from
exploiting these vulnerabilities, In this chapter, we will present the design of
the test suite for type confusion vulnerabilities, which decisions or language
specifications we want to evaluate the sanitizer on.

\section{Benchmarking Type Confusion}

\subsection{Initialization}

\noindent{}Initialization is a highly important step in a program. When creating 
an object, its type is assigned. At this point, both methods act on
values to keep track of their types, type++ by adding a vtable pointer to every
object and Hextype by recording all necessary information in its disjoint
metadata. We will evaluate their completeness i.e., their capacity to keep track 
and detect type confusion over a maximum of initialization methods. 
In the following, we will present various specific forms of object initialization 
on which we will evaluate the sanitizers.

\textbf{\href{https://en.cppreference.com/w/cpp/language/direct_initialization}{Direct
initialization}}

\noindent{}Direct initialization is the simplest way to initialize an object and
thus essential to measure if sanitizers efficiency. It consists of a creating an
object from a set of arguments.  In fact in Listing
\autoref{lst:type_confusion}, \texttt{password} is directly initialized with the
string "MyHardPassword" at line \ref{code:password}.

\textbf{\href{https://en.cppreference.com/w/cpp/language/copy_initialization}{Copy
initialization}}

\noindent{}Copy initialization is another way to initialize an object. 
It consists of creating an object from another object. 
With Copy initialization, we want to ensure that a sanitizer keep track of the 
type of object when it is copied. For instance, if we copy an
object of a derived class, we expect methods to keep track of the type of this object
and not lose any information, especially when casting to another type.

\textbf{\href{https://en.cppreference.com/w/cpp/language/list_initialization}{List
initialization}}

\noindent{}List initialization, directly or copy initialized, is a way to
initialize an object from a list of arguments.  It is a more recent way as it appears with 
C++11 to initialize an object and is particularly worthwhile to evaluate the completeness
of the sanitizer.

\textbf{C-style Allocators and \texttt{new} operator}

\noindent{}C-style allocators such as \texttt{malloc}, \texttt{calloc} and \texttt{realloc}
are used to allocate memory. They are a heritage from C and are still used in C++.
The \texttt{new} operator is a C++ operator that allocates memory and initializes the object.
These operators allocate memory dynamically i.e., at runtime and without a fixed size.
Unlike other methods which a return an object of a known type,
they return a void pointer that can be cast to any type. 
Depending on the size of the memory allocated, it can contain
multiple objects.  Thus, it is interesting to observe how the sanitizer behaves
when the memory is allocated with these functions and if they manage to keep
track correctly of the type of the objects.

\subsection{Modification \& Deletion}

\noindent{}Many edge cases happen when modifying the pointers by casting,
modifying the object referenced to the pointer, or deleting a pointer and
reusing it to observe or in order to study the limitations of sanitizers. For
instance, \texttt{placement new} is a variation of \texttt{new} operator and 
allows the programmer to construct an
object at already allocated memory.  Within a same memory block, we can
construct multiple objects of different types, which can lead to type confusion
if these objects are not correctly initialized. Furthermore, we try to modify the
pointers by casting, modifying the object referenced to, or deleting a pointer
and reusing it to observe if it correctly detects type confusion. These examples
are particularly known to be an issue for Hextype.
\subsection{Special cases}

Along with the initialization and modification of objects, there are some 
more precise cases that are interesting to evaluate because they are 
tricky to handle for a sanitizer. Next, we present some of these cases that 
generates type confusion vulnerabilities:

\textbf{Multiple Inheritance}

\noindent{}Multiple inheritance is a feature of C++ that allows a class to
inherit attributes and methods from multiple classes. It allows the creation of
complex class hierarchies and does not restrict the programmer to a single
inheritance per object. Furthermore, its application is truly advantageous when
objects have to share multiple characteristics of several type hierarchies. 

\textbf{Union}

\noindent{}A union is a special class type that allows to store different types
of data in the same memory location The feature is particularly useful for
saving memory as only the last object assigned to it is actually stored, but can
be risky to use and understand. When dealing with unions from the point of
view of type confusion, we must not confuse type confusion with undefined
behavior. In fact, it is up to the programmer to ensure that
the type of the value retrieve from the union is the correct one. With union type
confusions happen when the value retrieved is correct but is wrongly cast to
another type. For instance in Listing \autoref{lst:union}, the pointer
\texttt{u.b} references a \texttt{Base} object but is cast to a \texttt{Derived}
object at line \ref{code:union_type_confusion}.  This is a real type confusion and
should be detected by the sanitizer. However, at line
\ref{code:union_undefined_behavior}, the pointer \texttt{u.u} references the
same object as \texttt{u.b}(due to union nature) but is cast to an
\texttt{Unrelated} object.  Trying to access \texttt{u.u} will lead to undefined
behavior but is not a type confusion. Nevertheless, there exist tagged unions,
also known as \texttt{variant}, from C++17 onwards, which are a safer union that
uses a tag to keep track of the type of the object stored in the union. It will
not be evaluated in this paper but could be interesting to evaluate in the future.


\begin{listing}
       \begin{minted}[frame=single,linenos, breaklines, escapeinside=||]{C++}
class Base {};

class Derived : public Base {float x = 1.0;};

class Unrelated { public: void doSomething() {}};

union Union {
    Base *b;
    Unrelated *u;
};

int main () {
    Union u;
    u.b = new Base(); 
    Derived* d = static_cast<Derived*>(u.b); // Type confusion |\label{code:union_type_confusion}|
    u.u->doSomething(); // Undefined behavior |\label{code:union_undefined_behavior}|
    return 0;
}
       \end{minted}
       \caption{Union example}
       \label{lst:union}
\end{listing}

\textbf{Phantom Casting}

\noindent{}Phantom casting occurs when a base class and a derived have the same
data layout.  In this case, it is usually possible to cast a pointer from a base class
to a derived class even if the object is not of the derived class which should
be classified as a type confusion.  This special case exists because of backward
compatibility of C++ with C. Listing \autoref{lst:phantom_casting} shows an
example of phantom casting at line \ref{code:phantom_casting}.  The pointer
\texttt{base} references a \texttt{Base} object but is cast to a
\texttt{Derived} object, but this is allowed.

\begin{listing}
       \begin{minted}[frame=single,linenos, breaklines, escapeinside=||]{C++}
class Base {};

class Derived : public Base {};

int main(){
       Base* base = new Base();
       Derived* derived = static_cast<Derived*>(base); // Phantom casting |\label{code:phantom_casting}|
       return 0;
} 
       \end{minted}
       \caption{Phantom casting example}
       \label{lst:phantom_casting}

\end{listing}

%%%%%%%%%%%%%%%%%%%%%%%%
\chapter{Implementation}
%%%%%%%%%%%%%%%%%%%%%%%%

The implementation covers some of the implementation details of your project.
This is not intended to be a low level description of every line of code that
you wrote but covers the implementation aspects of the projects.

This section is usually 3-5 pages.

How are we going to implement the test suite according to what was previously defined? What are the challenges? 
How are we going to measure the results?

\begin{itemize}
       \item How are we going to implement the test suite according to what was previously defined? 
       \item What were the challenges?
       \item How are we going to measure the results?
       \item Example of test cases
\end{itemize}

We want to compare the results obtained from Hextype and type++ 
between them. The benchmark quality resides in the diversity and coverage
offered by the tests. In the following we will concentrate on C++11,
although there are more recent versions with more features dedicated to OOP. In
addition, we mainly test the standard C++11 without any external libraries or
frameworks to simplify and evaluate the basis of the language itself. 
Next, we would describe some challenge we faced during the implementation of the test suite.

\textbf{Set up sanitizers}

\noindent{}Setting up sanitizers was the first step in the implementation of the
test suite.  The test suite is designed to be run with Hextype and type++ to
evaluate their completeness on type confusion vulnerabilities.  It is located
inside type++ architecture and did not need to be installed. However, Hextype
had to be installed and configured to be able to run the test suite. Hextype can
only run up to C++14 and is not compatible with C++17 or C++20, which is a
limitation for the test suite. Hextype only triggers derived type confusion
whereas type++ triggers derived and unrelated type confusion inside and outside.
Furthermore, Hextype should update its type table to reflect the new type of the object, 
when a pointer is modified or used after being deleted, but sometimes it fails to do
so and produces false negatives or false positives. 


\textbf{Correctly trigger type confusion}

\noindent{}One challenge was to correctly create our test cases such that we
create legitimate type confusion.  Unions, phantom casting and multiple
inheritance were difficult to implement so that they generate legitimate type
confusion and not undefined behavior or valid cast. For instance, making simple
test cases by having very simple base class and a derived class could trigger
phantom casting and hence no type confusion were spotted. On another hand, we
spent a lot of time doing test cases on unions before realizing how to properly
create confused types and not undefined behaviors.  In order to reproduce the
test suite, it is important to have a class structure that is not too complex
but not too simple either, such that it does not trigger phantom casting or
multiple inheritance and treating those cases separately.

\begin{listing}
       \begin{minted}[frame=single,linenos, breaklines, escapeinside=||]{C++}

class A{};

class B: public A{
    int b = 3;
};

class C: public B{
    double c = 4.5;
}; 

int main(){
    A* a = new A();
    A* abis = new B();
    A* ac = new C();

    B* b = static_cast<B*>(a); // This is not allowed
    B* b2 = static_cast<B*>(abis); // This is allowed
    B* b3 = static_cast<B*>(ac); // This is allowed

    C* c = static_cast<C*>(a); // This is not allowed
    C* c2 = static_cast<C*>(abis); // This is not allowed
    C* c3 = static_cast<C*>(ac); // This is allowed
    C* c4 = static_cast<C*>(b2); // This is not allowed
    C* c5 = static_cast<C*>(b3); // This is allowed
}
       \end{minted}
       \caption{Example of test cases (taken from derived\_simple\_cast.cpp)}
       \label{lst:test_cases}
\end{listing}

%%%%%%%%%%%%%%%%%%%%
\chapter{Evaluation}
%%%%%%%%%%%%%%%%%%%%

In the evaluation you convince the reader that your design works as intended.
Describe the evaluation setup, the designed experiments, and how the
experiments showcase the individual points you want to prove.

This section is usually 5-10 pages.

Evalutates the test suite, explain the results
Stats that could be useful for a single sanitizers:
\begin{itemize}
       \item Correctness: Number of type confusion detected and classified as type confusion by the sanitizer while they were
       \item False positive: Number of type confusion detected and classified as type confusion by the sanitizer while they weren't ones
       \item False negative: Number of type confusion missed by the sanitizer i.e., classified as not type confusion but they were

       \item type confusion detected and well classified by both
       \item type confusion missed by one of the two
       \item type confusion missed by both

       \item Constraint and limit of the test suite

\end{itemize}

Give intuition on how to improve the current mechanisms

Later on,
it could be interesting to extend the test suite to more recent versions of C++
and to external library. 

%%%%%%%%%%%%%%%%%%%%%%
\chapter{Related Work}
%%%%%%%%%%%%%%%%%%%%%%

The related work section covers closely related work. Here you can highlight
the related work, how it solved the problem, and why it solved a different
problem. Do not play down the importance of related work, all of these
systems have been published and evaluated! Say what is different and how
you overcome some of the weaknesses of related work by discussing the 
trade-offs. Stay positive!

This section is usually 3-5 pages.

Discuss other sanitizers, other mechanisms to type confusion,
which may implements other mechanisms or same mechanisms but differently.

If time allows, test other sanitizer




%%%%%%%%%%%%%%%%%%%%
\chapter{Conclusion}
%%%%%%%%%%%%%%%%%%%%

In the conclusion you repeat the main result and finalize the discussion of
your project. Mention the core results and why as well as how your system
advances the status quo.

Concludes the project 
\cleardoublepage
\phantomsection
\addcontentsline{toc}{chapter}{Bibliography}
\printbibliography

% Appendices are optional
% \appendix
% %%%%%%%%%%%%%%%%%%%%%%%%%%%%%%%%%%%%%%
% \chapter{How to make a transmogrifier}
% %%%%%%%%%%%%%%%%%%%%%%%%%%%%%%%%%%%%%%
%
% In case you ever need an (optional) appendix.
%
% You need the following items:
% \begin{itemize}
% \item A box
% \item Crayons
% \item A self-aware 5-year old
% \end{itemize}

\end{document}
